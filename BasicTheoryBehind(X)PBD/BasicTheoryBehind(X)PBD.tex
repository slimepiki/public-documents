%Version 3 October 2023


%%\documentclass[referee,sn-basic]{sn-jnl}% referee option is meant for double line spacing

%%=======================================================%%
%% to print line numbers in the margin use lineno option %%
%%=======================================================%%

%%\documentclass[lineno,sn-basic]{sn-jnl}% Basic Springer Nature Reference Style/Chemistry Reference Style

%%======================================================%%
%% to compile with pdflatex/xelatex use pdflatex option %%
%%======================================================%%

%%=============================================================%%
%% GivenName	-> \fnm{Joergen W.}
%% Particle	-> \spfx{van der} -> surname prefix
%% FamilyName	-> \sur{Ploeg}
%% Suffix	-> \sfx{IV}
%% \author*[1,2]{\fnm{Joergen W.} \spfx{van der} \sur{Ploeg} 
%%  \sfx{IV}}\email{iauthor@gmail.com}
%%=============================================================%%


%%==================================%%
%% Sample for unstructured abstract %%%\author*[1,2]{\fnm{First} \sur{Author}}\email{iauthor@gmail.com}
%
%\author[2,3]{\fnm{Second} \sur{Author}}\email{iiauthor@gmail.com}
%\equalcont{These authors contributed equally to this work.}
%
%\author[1,2]{\fnm{Third} \sur{Author}}\email{iiiauthor@gmail.com}
%\equalcont{These authors contributed equally to this work.}
%
%\affil*[1]{\orgdiv{Department}, \orgname{Organization}, \orgaddress{\street{Street}, \city{City}, \postcode{100190}, \state{State}, \country{Country}}}
%%==================================%%

%\abstract{The abstract serves both as a general introduction to the topic and as a brief, non-technical summary of the main results and their implications. Authors are advised to check the author instructions for the journal they are submitting to for word limits and if structural elements like subheadings, citations, or equations are permitted.}

%%================================%%
%% Sample for structured abstract %%
%%================================%%

% \abstract{\textbf{Purpose:} 
% 
% \textbf{Methods:}
% 
% \textbf{Results:} 
% 
% \textbf{Conclusion:} }

%\keywords{keyword1, Keyword2, Keyword3, Keyword4}

%%\pacs[JEL Classification]{D8, H51}

%%\pacs[MSC Classification]{35A01, 65L10, 65L12, 65L20, 65L70}

% Reference Style/Chemistry Reference Style


%%\documentclass[sn-nature]{sn-jnl}% Style for submissions to Nature Portfolio journals
%%\documentclass[sn-basic]{sn-jnl}% Basic Springer Nature Reference Style/Chemistry Reference Style
%%\documentclass[sn-mathphys-num]{sn-jnl}% Math and Physical Sciences Numbered Reference Style 
%%\documentclass[sn-mathphys-ay]{sn-jnl}% Math and Physical Sciences Author Year Reference Style
%%\documentclass[sn-aps]{sn-jnl}% American Physical Society (APS) Reference Style
%%\documentclass[sn-vancouver,Numbered]{sn-jnl}% Vancouver Reference Style
%%\documentclass[sn-apa]{sn-jnl}% APA Reference Style 
%%\documentclass[sn-chicago]{sn-jnl}% Chicago-based Humanities Reference Style

%%%% Standard Packages
%%<additional latex packages if required can be included here>



%%%%%=============================================================================%%%%
%%%%  Remarks: This template is provided to aid authors with the preparation
%%%%  of original research articles intended for submission to journals published 
%%%%  by Springer Nature. The guidance has been prepared in partnership with 
%%%%  production teams to conform to Springer Nature technical requirements. 
%%%%  Editorial and presentation requirements differ among journal portfolios and 
%%%%  research disciplines. You may find sections in this template are irrelevant 
%%%%  to your work and are empowered to omit any such section if allowed by the 
%%%%  journal you intend to submit to. The submission guidelines and policies 
%%%%  of the journal take precedence. A detailed User Manual is available in the 
%%%%  template package for technical guidance.
%%%%%=============================================================================%%%%

\documentclass[pdflatex,sn-mathphys-num]{sn-jnl}% Basic Springer Nature
\usepackage{graphicx}%
\usepackage{multirow}%
\usepackage{amsmath,amssymb,amsfonts}%
\usepackage{amsthm}%
\usepackage{mathrsfs}%
\usepackage[title]{appendix}%
\usepackage{xcolor}%
\usepackage{textcomp}%
\usepackage{manyfoot}%
\usepackage{booktabs}%
\usepackage{algorithm}%
\usepackage{algorithmicx}%
\usepackage{algpseudocode}%
\usepackage{listings}%
\usepackage{glossaries}

%% as per the requirement new theorem styles can be included as shown below
\theoremstyle{thmstyleone}%
\newtheorem{theorem}{Theorem}%  meant for continuous numbers
%%\newtheorem{theorem}{Theorem}[section]% meant for sectionwise numbers
%% optional argument [theorem] produces theorem numbering sequence instead of independent numbers for Proposition
\newtheorem{proposition}[theorem]{Proposition}% 
%%\newtheorem{proposition}{Proposition}% to get separate numbers for theorem and proposition etc.

\theoremstyle{thmstyletwo}%
\newtheorem{example}{Example}%
\newtheorem{remark}{Remark}%

\theoremstyle{thmstylethree}%
\newtheorem{definition}{Definition}%
\makeglossaries
\newcommand{\pname}[1]{``{\sl #1}''}

\newglossaryentry{iteration}
{
	name=iteration,
	description={In computer science, iteration is the process of repeating a series of instructions multiple times. Especially at (X)PBD, the part of physical solver which treats \gls{constraint}s is called iteration.}
}

\newglossaryentry{constraint}{
	name = constraint,
	description={
		At the (X)PBD, constraint is relationship among arbitrary vertices what we want to keep during simulation.
		It can express not only between two vertices relation, e.g. distance a vertex to another one, but among more than three vertices, e.g. volume of the tetrahedron.
	}
}
%%%%
\raggedbottom
%%\unnumbered% uncomment this for unnumbered level heads
\begin{document}
	
	\title[Basic theory behind (X)PBD]{Basic theory behind (X)PBD}
	\author*{\fnm{Slime} \sur{Piki}}
\maketitle
\tableofcontents
\newpage

\section{Introduction}\label{sec1}
\subsection{What is (X)PBD?}
PBD (Position Based Dynamics) proposed at \cite{PBD} is a popular method because of its stability and ease of implementation. The reason for them is the same, PBD computes physical simulation only using positions inside the \gls{iteration}s and all we have to do is compute displacement and modify them.
In other words, we don't have to use complicated numerical analysis, it sounds pretty good.

But, in contrast to ease of implementation, it isn't easy to understand PBD's background theory. This is the problem when modifying PBD depending on your purpose. 

If you start your research from the original PBD paper\cite{PBD}, you will wonder how the authors derive constraints' formulations or why this solver works well. Or you start from XPBD \cite{XPBD}, you will be confused by the suddenly appeared Lagrange multiplier or energy potential that we don't know how to handle.
Unfortunately, we can't know much from them and it may be common in literature search, there is no clear path to learning them. Then, I decided to write a guidebook on the underlying theory of PBD.

\subsection{Difference from existing PBD coursenote}
Actually, there are some course notes on PBD written by authors who published papers on PBD and XPBD, e.g. \cite{PBDCoursenote}. These course notes describe the basic style of PBD and its extensions. But there is the same problem we saw in \cite{PBD} and \cite{XPBD}, that is, how to implement is described but why this method works well is not. Thus, I believe that this document isn't meaningless. Well then, let's start the journey to XPBD!

\section{The history of PBD}
I think starting from history is a good way to learn something because there are no leaps in logic and it will be easy to understand where we are. However, there is certainly redundancy, so you can skip this section to save time. I'll make an effort to write that you can understand everything even if you skipped this section.
\subsection{PBD's chronicle}
The history of PBD can be roughly divided into three parts.
Let me name them pre-PBD, post-PBD and post-XPBD.

\subsection{pre-PBD}
The general flow of pre-PBD began from the appearance of constraint dynamics(\cite{EnergyWitkin1987}, \cite{ConstrainedBarzel} and \cite{ConstrainedPlatt}) through \pname{Large Steps in Cloth Simulation}\cite{LargeStepBaraff} simulates cloth with constraints, \pname{Advanced Character Physics}\cite{Jakobsen2003AdvancedCP} introduces position-based approach with the distance constraint and \pname{A Versatile and Robust Model for Geometrically Complex Deformable Solids}\cite{VersatileTeschner} derives several constraints as energy functions.
And, finally, with these benefits, \pname{Position Based Dynamics}\cite{PBD} having the advantage I mentioned above was published.

\subsubsection{Pioneer scholars}

	\begin{itemize}
	\item Andrew Witkin
	\item David Barraff
	\item Alan Barr
	\item Ronen Barzel
	\item Jhon Platt
	\item Matthias Müller
	\item Matthias Teschner
\end{itemize}

\subsection{post-PBD}

\subsubsection{Pioneer scholars}

\begin{itemize}
	\item Matthias Müller
	
	\item Nuttapong Chentanez
\end{itemize}

\subsection{post-XPBD}
\newpage



\subsubsection{Pioneer scholars}

\begin{itemize}
	\item Miles Macklin
\end{itemize}

\appendix




\section{Inportance of papers}
A lower number means more important.
\begin{enumerate}
	\item You must read these papers. But if you want to understand them completely, I recommend reading the others also.
		\begin{itemize}
			\item \pname{Position Based Dynamics}\cite{PBD} is one of the main subjects of this document.
			\item \pname{XPBD: position-based simulation of compliant constrained dynamics}\cite{XPBD}  addressed a numerical artifact that makes dependency between stiffness and \gls{iteration} count or size of the time-step.
		\end{itemize}
	\item  These papers are vital in understanding (X)PBD or have a strong impact on the field.
		\begin{itemize}
			\item 
		\end{itemize}
	\item These papers offer interesting discussions around PBD or deepen your understanding of PBD.
		\begin{itemize}
			\item 
		\end{itemize}
	\item These papers have historical value, but deeper discussions are done in other papers.
		\begin{itemize}
		\item \pname{Elastically deformable models}\cite{ElasticTerzopoulos}  brought the formulation of elastic bodies to computer graphics.
		\item \pname{Energy Constraints On Parameterized Models}\cite{Witkin1987}The shape representations presented in this paper are quite different from the current ones and the constraints presented in this paper are slightly inconvenient to the current ones. Thus, we no longer have to read this.
		\end{itemize}
	\item Not be classified yet.
		\begin{itemize}
			
			\item \pname{A modeling system based on dynamic constraints}\cite{Barzel1988}
			\item \pname{Constraint Methods for Neural Networks and Computer Graphics}\cite{ConstrainedPlatt}
			\item \pname{Large steps in cloth simulation}\cite{LargeStepBaraff}
			\item \pname{Advanced Character Physics}\cite{Jakobsen2003AdvancedCP}
			\item \pname{A Versatile and Robust Model for Geometrically Complex Deformable Solids}\cite{VersatileTeschner}
			\item \pname{Meshless deformations based on shape matching}\cite{MeshlessMuller2005}
			\item \pname{Interactive simulation of elastic deformable materials}\cite{Servin2006InteractiveSO}
			\item \pname{Efficient simulation of inextensible cloth}\cite{EfficientGoldenthai}
			\item \pname{Fast Simulation of Inextensible Hair and Fur}\cite{InextensibleHair}
			\item \pname{Long Range Attachments - A Method to Simulate Inextensible Clothing in Computer Games}\cite{LongRangeAttachments}
			\item \pname{Position Based Fluids}\cite{PosBaseFluids}
			\item \pname{Position-based simulation of continuous materials}\cite{PBContimuousBENDER2014}
			\item \pname{Strain Based Dynamics}\cite{StrainBasedDyn}
			\item \pname{Unified particle physics for real-time applications}\cite{UnifiedParticle}
			\item \pname{Air Meshes for Robust Collision Handling}\cite{AirMesh}
			\item \pname{A survey on position based dynamics, 2017}\cite{PBDCoursenote}
			\item \pname{Stable Constrained Dynamics}\cite{StableConstrainedDyn}
			\item \pname{Small steps in physics simulation}\cite{SmallSteps}
			\item \pname{Non-Smooth Newton Methods for Deformable Multi-Body Dynamics}\cite{NonSmooth}
			\item \pname{Detailed Rigid Body Simulation with Extended Position Based Dynamics}\cite{DetailedRigidbody}
			\item \pname{A Constraint-based Formulation of Stable Neo-Hookean Materials}\cite{ConstrainedBasedNeo-Hookean}
			\item \pname{Physically Based Shape Matching}\cite{PhysBaseShapeMatching}
			\item \pname{}\cite{}
			\item \pname{}\cite{}
			\item \pname{}\cite{}
			\item \pname{}\cite{}
			\item \pname{}\cite{}
			\item \pname{}\cite{}
			\item \pname{}\cite{}
			\item \pname{}\cite{}
			\item \pname{}\cite{}
			\item \pname{}\cite{}
			\item \pname{}\cite{}
			\item \pname{}\cite{}
			\item \pname{}\cite{}
			\item \pname{}\cite{}
			\item \pname{}\cite{}
			\item \pname{}\cite{}
			\item \pname{}\cite{}
			\item \pname{}\cite{}
			\item \pname{}\cite{}
			\item \pname{}\cite{}
			\item \pname{}\cite{}
			\item \pname{}\cite{}
			\item \pname{}\cite{}
			\item \pname{}\cite{}
			\item \pname{}\cite{}
			\item \pname{}\cite{}
			\item \pname{}\cite{}
		\end{itemize}
	
\end{enumerate}


\section{Glossaly}\label{ap1}
\subsection{symbols}

\subsection{terms}
\printglossaries

\appendix

\addcontentsline{toc}{section}{References}
\bibliography{sn-bibliography}% common bib file


\end{document}


% \cite{bib1} 

%%==================%%
%% equation samples %%
%%==================%%
%\section{Equations}\label{sec4}
%$H\psi = E \psi$ is written via the command \verb+$H \psi = E \psi$+.

%\begin{equation}
%\|\tilde{X}(k)\|^2 \leq\frac{\sum\limits_{i=1}^{p}\left\|\tilde{Y}_i(k)\right\|^2+\sum\limits_{j=1}^{q}\left\|\tilde{Z}_j(k)\right\|^2 }{p+q}.\label{eq1}
%\end{equation}

%where,

%\begin{align}
%D_\mu &=  \partial_\mu - ig \frac{\lambda^a}{2} A^a_\mu \nonumber \\
%F^a_{\mu\nu} &= \partial_\mu A^a_\nu - \partial_\nu A^a_\mu + g f^{abc} A^b_\mu A^a_\nu \label{eq2}
%\end{align}

%\begin{equation}
%Y_\infty = \left( \frac{m}{\textrm{GeV}} \right)^{-3}
%    \left[ 1 + \frac{3 \ln(m/\textrm{GeV})}{15}
%    + \frac{\ln(c_2/5)}{15} \right]
%\end{equation}


%%==============%%
%%Table samples %%
%%==============%%

%\begin{table}[h]
%\caption{Caption text}\label{tab1}%
%\begin{tabular}{@{}llll@{}}
%\toprule
%Column 1 & Column 2  & Column 3 & Column 4\\
%\midrule
%row 1    & data 1   & data 2  & data 3  \\
%row 2    & data 4   & data 5\footnotemark[1]  & data 6  \\
%row 3    & data 7   & data 8  & data 9\footnotemark[2]  \\
%\botrule
%\end{tabular}
%\footnotetext{Source: This is an example of table footnote. This is an example of table footnote.}
%\footnotetext[1]{Example for a first table footnote. This is an example of table footnote.}
%\footnotetext[2]{Example for a second table footnote. This is an example of table footnote.}
%\end{table}

%===========================================================

%\begin{verbatim}
%\begin{table}[<placement-specifier>]
%\caption{<table-caption>}\label{<table-label>}%
%\begin{tabular}{@{}llll@{}}
%\toprule
%Column 1 & Column 2 & Column 3 & Column 4\\
%\midrule
%row 1 & data 1 & data 2	 & data 3 \\
%row 2 & data 4 & data 5\footnotemark[1] & data 6 \\
%row 3 & data 7 & data 8	 & data 9\footnotemark[2]\\
%\botrule
%\end{tabular}
%\footnotetext{Source: This is an example of table footnote. 
%This is an example of table footnote.}
%\footnotetext[1]{Example for a first table footnote.
%This is an example of table footnote.}
%\footnotetext[2]{Example for a second table footnote. 
%This is an example of table footnote.}
%\end{table}
%\end{verbatim}

%===========================================================

%\begin{table}[h]
%\caption{Example of a lengthy table which is set to full textwidth}\label{tab2}
%\begin{tabular*}{\textwidth}{@{\extracolsep\fill}lcccccc}
%\toprule%
%& \multicolumn{3}{@{}c@{}}{Element 1\footnotemark[1]} & \multicolumn{3}{@{}c@{}}{Element 2\footnotemark[2]} \\\cmidrule{2-4}\cmidrule{5-7}%
%Project & Energy & $\sigma_{calc}$ & $\sigma_{expt}$ & Energy & $\sigma_{calc}$ & $\sigma_{expt}$ \\
%\midrule
%Element 3  & 990 A & 1168 & $1547\pm12$ & 780 A & 1166 & $1239\pm100$\\
%Element 4  & 500 A & 961  & $922\pm10$  & 900 A & 1268 & $1092\pm40$\\
%\botrule
%\end{tabular*}
%\footnotetext{Note: This is an example of table footnote. This is an example of table footnote this is an example of table footnote this is an example of~table footnote this is an example of table footnote.}
%\footnotetext[1]{Example for a first table footnote.}
%\footnotetext[2]{Example for a second table footnote.}
%\end{table}

%===========================================================

%\begin{sidewaystable}
%\caption{Tables which are too long to fit, should be written using the ``sidewaystable'' environment as shown here}\label{tab3}
%\begin{tabular*}{\textheight}{@{\extracolsep\fill}lcccccc}
%\toprule%
%& \multicolumn{3}{@{}c@{}}{Element 1\footnotemark[1]}& \multicolumn{3}{@{}c@{}}{Element\footnotemark[2]} \\\cmidrule{2-4}\cmidrule{5-7}%
%Projectile & Energy	& $\sigma_{calc}$ & $\sigma_{expt}$ & Energy & $\sigma_{calc}$ & $\sigma_{expt}$ \\
%\midrule
%Element 3 & 990 A & 1168 & $1547\pm12$ & 780 A & 1166 & $1239\pm100$ \\
%Element 4 & 500 A & 961  & $922\pm10$  & 900 A & 1268 & $1092\pm40$ \\
%Element 5 & 990 A & 1168 & $1547\pm12$ & 780 A & 1166 & $1239\pm100$ \\
%Element 6 & 500 A & 961  & $922\pm10$  & 900 A & 1268 & $1092\pm40$ \\
%\botrule
%\end{tabular*}
%\footnotetext{Note: This is an example of table footnote this is an example of table footnote this is an example of table footnote this is an example of~table footnote this is an example of table footnote.}
%\footnotetext[1]{This is an example of table footnote.}
%\end{sidewaystable}


%%================%%
%%a figure sample %%
%%================%%
%\begin{figure}[h]
%\centering
%\includegraphics[width=0.9\textwidth]{fig.eps}
%\caption{This is a widefig. This is an example of long caption this is an example of long caption  this is an example of long caption this is an example of long caption}\label{fig1}
%\end{figure}


%%==============================================%%
%%Algorithms, Program codes and Listings samples%%
%%==============================================%%


%\lstset{texcl=true,basicstyle=\small\sf,commentstyle=\small\rm,mathescape=true,escapeinside={(*}{*)}}
%\begin{lstlisting}
%begin
%  for $i:=1$ to $10$ step $1$ do
%      expt($2,i$);  
%      newline() od                (*\textrm{Comments will be set flush to the right margin}*)
%where
%proc expt($x,n$) $\equiv$
%  $z:=1$;
%  do if $n=0$ then exit fi;
%     do if odd($n$) then exit fi;                 
%        comment: (*\textrm{This is a comment statement;}*)
%        $n:=n/2$; $x:=x*x$ od;
%     { $n>0$ };
%     $n:=n-1$; $z:=z*x$ od;
%  print($z$). 
%end
%\end{lstlisting}

%===========================================================

%
%\begin{algorithm}
%\caption{Calculate $y = x^n$}\label{algo1}
%\begin{algorithmic}[1]
%\Require $n \geq 0 \vee x \neq 0$
%\Ensure $y = x^n$ 
%\State $y \Leftarrow 1$
%\If{$n < 0$}\label{algln2}
%        \State $X \Leftarrow 1 / x$
%        \State $N \Leftarrow -n$
%\Else
%        \State $X \Leftarrow x$
%        \State $N \Leftarrow n$
%\EndIf
%\While{$N \neq 0$}
%        \If{$N$ is even}
%            \State $X \Leftarrow X \times X$
%            \State $N \Leftarrow N / 2$
%        \Else[$N$ is odd]
%            \State $y \Leftarrow y \times X$
%            \State $N \Leftarrow N - 1$
%        \EndIf
%\EndWhile
%\end{algorithmic}
%\end{algorithm}

%===========================================================

%\begin{minipage}{\hsize}%
%\lstset{frame=single,framexleftmargin=-1pt,framexrightmargin=-17pt,framesep=12pt,linewidth=0.98\textwidth,language=pascal}% Set your language (you can change the language for each code-block optionally)
%%%% Start your code-block
%\begin{lstlisting}
%for i:=maxint to 0 do
%begin
%{ do nothing }
%end;
%Write('Case insensitive ');
%Write('Pascal keywords.');
%\end{lstlisting}
%\end{minipage}

%%==================%%
%% citation samples %%
%%==================%%

%Here is an example for \verb+\cite{...}+: \cite{bib1}. Another example for \verb+\citep{...}+: \citep{bib2}. For author-year citation mode, \verb+\cite{...}+ prints Jones et al. (1990) and \verb+\citep{...}+ prints (Jones et al., 1990).

%===========================================================

%All cited bib entries are printed at the end of this article: \cite{bib3}, \cite{bib4}, \cite{bib5}, \cite{bib6}, \cite{bib7}, \cite{bib8}, \cite{bib9}, \cite{bib10}, \cite{bib11}, \cite{bib12} and \cite{bib13}.

%%====================%%
%% mathematic samples %%
%%====================%%

%\begin{tabular}{|l|p{19pc}|}
%\hline
%\verb+thmstyleone+ & Numbered, theorem head in bold font and theorem text in italic style \\\hline
%\verb+thmstyletwo+ & Numbered, theorem head in roman font and theorem text in italic style \\\hline
%\verb+thmstylethree+ & Numbered, theorem head in bold font and theorem text in roman style \\\hline
%\end{tabular}

%===========================================================

%\begin{theorem}[Theorem subhead]\label{thm1}
%\end{theorem}
%
%\begin{proposition}
%\end{proposition}
%
%\begin{example}
%\end{example}
%
%
%\begin{remark}
%\end{remark}
%
%\begin{definition}[Definition sub head]
%\end{definition}
%
%
%\begin{proof}
%\end{proof}
%
%
%\begin{proof}[Proof of Theorem~{\upshape\ref{thm1}}]
%\end{proof}
%
%\begin{quote}
%\end{quote}

 %(\url{https://www.nature.com/nature-research/editorial-policies}) for Nature Portfolio journals, 

\backmatter

%\bmhead{Supplementary information}
%If your article 

