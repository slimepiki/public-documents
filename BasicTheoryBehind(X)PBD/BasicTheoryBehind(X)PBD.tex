%Version 3 October 2023


%%\documentclass[referee,sn-basic]{sn-jnl}% referee option is meant for double line spacing

%%=======================================================%%
%% to print line numbers in the margin use lineno option %%
%%=======================================================%%

%%\documentclass[lineno,sn-basic]{sn-jnl}% Basic Springer Nature Reference Style/Chemistry Reference Style

%%======================================================%%
%% to compile with pdflatex/xelatex use pdflatex option %%
%%======================================================%%

%%=============================================================%%
%% GivenName	-> \fnm{Joergen W.}
%% Particle	-> \spfx{van der} -> surname prefix
%% FamilyName	-> \sur{Ploeg}
%% Suffix	-> \sfx{IV}
%% \author*[1,2]{\fnm{Joergen W.} \spfx{van der} \sur{Ploeg} 
%%  \sfx{IV}}\email{iauthor@gmail.com}
%%=============================================================%%


%%==================================%%
%% Sample for unstructured abstract %%%\author*[1,2]{\fnm{First} \sur{Author}}\email{iauthor@gmail.com}
%
%\author[2,3]{\fnm{Second} \sur{Author}}\email{iiauthor@gmail.com}
%\equalcont{These authors contributed equally to this work.}
%
%\author[1,2]{\fnm{Third} \sur{Author}}\email{iiiauthor@gmail.com}
%\equalcont{These authors contributed equally to this work.}
%
%\affil*[1]{\orgdiv{Department}, \orgname{Organization}, \orgaddress{\street{Street}, \city{City}, \postcode{100190}, \state{State}, \country{Country}}}
%%==================================%%

%\abstract{The abstract serves both as a general introduction to the topic and as a brief, non-technical summary of the main results and their implications. Authors are advised to check the author instructions for the journal they are submitting to for word limits and if structural elements like subheadings, citations, or equations are permitted.}

%%================================%%
%% Sample for structured abstract %%
%%================================%%

% \abstract{\textbf{Purpose:} 
% 
% \textbf{Methods:}
% 
% \textbf{Results:} 
% 
% \textbf{Conclusion:} }

%\keywords{keyword1, Keyword2, Keyword3, Keyword4}

%%\pacs[JEL Classification]{D8, H51}

%%\pacs[MSC Classification]{35A01, 65L10, 65L12, 65L20, 65L70}

% Reference Style/Chemistry Reference Style


%%\documentclass[sn-nature]{sn-jnl}% Style for submissions to Nature Portfolio journals
%%\documentclass[sn-basic]{sn-jnl}% Basic Springer Nature Reference Style/Chemistry Reference Style
%%\documentclass[sn-mathphys-num]{sn-jnl}% Math and Physical Sciences Numbered Reference Style 
%%\documentclass[sn-mathphys-ay]{sn-jnl}% Math and Physical Sciences Author Year Reference Style
%%\documentclass[sn-aps]{sn-jnl}% American Physical Society (APS) Reference Style
%%\documentclass[sn-vancouver,Numbered]{sn-jnl}% Vancouver Reference Style
%%\documentclass[sn-apa]{sn-jnl}% APA Reference Style 
%%\documentclass[sn-chicago]{sn-jnl}% Chicago-based Humanities Reference Style

%%%% Standard Packages
%%<additional latex packages if required can be included here>



%%%%%=============================================================================%%%%
%%%%  Remarks: This template is provided to aid authors with the preparation
%%%%  of original research articles intended for submission to journals published 
%%%%  by Springer Nature. The guidance has been prepared in partnership with 
%%%%  production teams to conform to Springer Nature technical requirements. 
%%%%  Editorial and presentation requirements differ among journal portfolios and 
%%%%  research disciplines. You may find sections in this template are irrelevant 
%%%%  to your work and are empowered to omit any such section if allowed by the 
%%%%  journal you intend to submit to. The submission guidelines and policies 
%%%%  of the journal take precedence. A detailed User Manual is available in the 
%%%%  template package for technical guidance.
%%%%%=============================================================================%%%%

\documentclass[pdflatex,sn-mathphys-num]{sn-jnl}% Basic Springer Nature
\usepackage{multirow}%
\usepackage{graphicx}
\usepackage{amsmath,amssymb,amsfonts}%
\usepackage{amsthm}%
\usepackage{mathrsfs}%
\usepackage[title]{appendix}%
\usepackage[dvipdfmx]{xcolor}%
\usepackage{textcomp}%
\usepackage{manyfoot}%
\usepackage{booktabs}%
\usepackage{algorithm}%
\usepackage{algorithmicx}%
\usepackage{algpseudocode}%
\usepackage{listings}%
\usepackage[xindy]{glossaries}
\usepackage{ulem}
\usepackage{bm}
\usepackage{caption}
\captionsetup[table]{justification=centering}
\captionsetup[figure]{justification=centering}
\renewcommand{\glossarysection}[2][]{}

%% as per the requirement new theorem styles can be included as shown below
\theoremstyle{thmstyleone}%
\newtheorem{theorem}{Theorem}%  meant for continuous numbers
%%\newtheorem{theorem}{Theorem}[section]% meant for sectionwise numbers
%% optional argument [theorem] produces theorem numbering sequence instead of independent numbers for Proposition
\newtheorem{proposition}[theorem]{Proposition}% 
%%\newtheorem{proposition}{Proposition}% to get separate numbers for theorem and proposition etc.

\theoremstyle{thmstyletwo}%
\newtheorem{example}{Example}%
\newtheorem{remark}{Remark}%

\theoremstyle{thmstylethree}%
\newtheorem{definition}{Definition}%
\makeglossaries
\setacronymstyle{long-short-desc}
\loadglsentries{glossary.tex}
\newcommand{\pname}[1]{``{\uline{\sl {#1}}}''}
\newcommand{\mainpname}[1]{``{\sl {#1}}''}
\newcommand{\inprep}{
	\begin{center}
		\sl\rm {!!!!In preparation!!!!}
\end{center}}

%%%%
\raggedbottom

%%\unnumbered% uncomment this for unnumbered level heads
\begin{document}
	\title[Basic theory behind (X)PBD]{Basic theory behind (X)PBD}
	\author*{\fnm{Slime} \sur{Piki}(Taisei Kunimi)}
\maketitle
\tableofcontents
\newpage

\section{Introduction}\label{sec1}
\subsection{What is (X)PBD?}
PBD is one of the simulation theories that basically simulates \gls{softBody} or \gls{elasticBody}.

PBD (Position Based Dynamics) proposed at \cite{PBD} is a popular method because of its stability and ease of implementation.In datail, PBD computes physical simulation only using positions inside the \gls{iteration}s and all we have to do is compute displacement and modify them.
In other words, we don't have to use complicated numerical analysis theories, it sounds pretty good.

But, in contrast to ease of implementation, it isn't easy to understand PBD's background theory. This is the problem when modifying PBD depending on your purpose. 

If you start your research from the original PBD paper\cite{PBD}, you will wonder how the authors derive constraints' formulations or why this solver works well. Or you start from XPBD \cite{XPBD}, you will be confused by the suddenly appeared Lagrange multiplier or energy potential that we don't know how to handle.
Unfortunately, we can't know much from them and it may be common in literature search, there is no clear path to learning them. Then, I decided to write a guidebook on the underlying theory of PBD.

\subsection{Difference from existing PBD coursenote}
Actually, there are some course notes on PBD written by authors who published papers on PBD and XPBD, e.g. \cite{PBDCoursenote}. These course notes describe the basic style of PBD and its extensions. But there is the same problem we saw in \cite{PBD} and \cite{XPBD}, that is, how to implement is described but why this method works well is not. Thus, I believe that this document isn't meaningless. Well then, let's start the journey to XPBD!

\subsection{Learning Path}
Fig. \ref{XPBDPath} shows the shortest path to understand (X)PBD. You don't have to read this note from head to tail because this note covers wide topics around (X)PBD. If you only want to learn how (X)PBD works, you read this note along with Fig. \ref{XPBDPath}'s path. In addition, the dashed circles could be skipped if you are in a hurry.

\begin{figure}[h]
	\centering
	\begin{minipage}[b]{0.49\columnwidth}
	\centering
\includegraphics[width=1.05\textwidth]{images/(X)PBD_path.png}
\caption{Path to (X)PBD}
\label{XPBDPath}
	\end{minipage}
	\begin{minipage}[b]{0.49\columnwidth}
	\centering
\includegraphics[width=1.1\textwidth]{images/ProjDyn_path.png}
\caption{Path to (X)PBD}
\label{ProjPath}
	\end{minipage}
\end{figure}

As the extra, you can use this note to learn Projective Dynamics also, yay!
If you want to do so, the shortest path is fig. \ref{ProjPath}.

\section{The history of PBD}
I think starting from history is a good way to learn something because there are no leaps in logic and it will be easy to understand where we are. However, there is certainly redundancy, so you can skip this section to save time. I'll make an effort to write that you can understand everything even if you skipped this section.

The history of PBD can be roughly divided into three parts.
Let them are pre-PBD, post-PBD and post-XPBD.

\subsection{Pre-PBD}
\begin{figure}[h]
\centering
\includegraphics[width=0.8\textwidth]{images/prePBD.png}
\caption{history of pre-PBD}
\label{prePBD}
\end{figure}

The flow of the pre-PBD began from the appearance of constraint dynamics(\cite{EnergyWitkin1987}, \cite{ConstrainedBarzel} and \cite{ConstrainedPlatt}) through \mainpname{Large Steps in Cloth Simulation}\cite{LargeStepBaraff} used \gls{constraint}s as shape representation and simulates cloth with energy form \gls{constraint}s' gradients, \mainpname{Advanced Character Physics}\cite{Jakobsen2003AdvancedCP} introduces position-based approach derived from Verlet's integration scheme with the distance constraints and \mainpname{A Versatile and Robust Model for Geometrically Complex Deformable Solids}\cite{VersatileTeschner} generalize \cite{LargeStepBaraff}'s method.
And, finally, \mainpname{Position Based Dynamics}\cite{PBD} introduced the generalized constraints method from \cite{VersatileTeschner} into \cite{Jakobsen2003AdvancedCP}'s position-based simulation to use various \gls{constraint}s.

\subsection{Post-PBD}
\begin{figure}[h]
	\centering
	\includegraphics[width=1\textwidth]{images/postPBD.png}
	\caption{history of post-PBD}
		\label{postPBD}
\end{figure}

From the published PBD paper, a lot of study about it has been done.
The central development is resolving well-known PBD's drawback whose result depends on \gls{iteration} times, at \mainpname{XPBD: position-based simulation of compliant constrained dynamics}\cite{XPBD}.

To understand XPBD, we need some knowledge of continuum mechanics.
The theory directly used in XPBD is derived from \mainpname{Interactive simulation of elastic deformable materials}\cite{Servin2006InteractiveSO} that uses the theory for \gls{LagrangianMechanics}.

Besides that, some papers also influence XPBD, \mainpname{Efficient simulation of inextensible cloth}\cite{EfficientGoldenthai} and \mainpname{Strain Based Dynamics}\cite{StrainBasedDyn} are examples of them.
The former introduces the projection method described later. The latter invites physically based constraints to PBD.

Although there is relevance to XPBD, the key idea to construct XPBD is \cite{Servin2006InteractiveSO} and aren't the last two papers. In other words, to understand XPBD, we truly only have to read \cite{Servin2006InteractiveSO} or get the equivalent knowledge.

But, unfortunately, the key idea of XPBD provided by \cite{Servin2006InteractiveSO} doesn't have a sufficient explanation of where the authors get the formulation. Then, I guess a description later that seemed reasonable enough.

\subsection{Post-XPBD}
Shortly after published XPBD paper, a simple but important improvement was provided at \mainpname{Small steps in physics simulation}\cite{SmallSteps}.
\inprep
\section{Physics}
This section is devoted to fundamental physics. 

Of course, the physics simulation field takes advantage of the basics.
If we only handle trivial environments, classical mechanics can achieve simulation. 

However, classical mechanics isn't enough to simulate complex objects or get plausible results. For example, Newtonian mechanics-based methods suffer from soft-body simulation, and so on.

The algorithm of the soft-body simulation isn't trivial, and there is no definitive solution at present. Therefore, many methods have been studied to solve this problem.

Based on the above perspective, in this section, I also introduce some physical theories that are frequently used in soft/elastic body simulation, Constrained dynamics, and Continuum mechanics.
In particular, continuum mechanics is vital to understanding XPBD.
\begin{figure}[h]
	\centering
	\includegraphics[width=0.8\textwidth]{images/Physics.png}
	\caption{Comparison of the physics}
		\label{physics}
\end{figure}

Although there are many other important theories, such as fluid mechanics or thermodynamics, I won't explain them because this document's subject is (X)PBD which focuses on soft/elastic body (actually, PBD can be used in fluid simulation...).

\subsection{Classical mechanics}
Let's start the physics course with classical mechanics.
\subsubsection{Position, Velocity and, Acceleration}
\begin{figure}[H]
	\centering
	\includegraphics[width=0.45\textwidth]{images/constCar.png}
	\caption{constantly moving car}
	\label{cscar}
\end{figure}
First, please imagine a vehicle running on a line at a constant speed.
Next, we measure the vehicle's position once per second and start your stopwatch with zero. From now on, time is measured in seconds and denoted as $t = (current time in seconds)$
Let the first measured position $x_0$, the second position $x_1$, and so on.

Because the vehicle has a constant speed, the distance it moves between a single step is the same. In other words, $x_{n}-x_{n-1}$ is an invariant for any natural integer $n$.

The moved distance per second is called velocity whose unit [$(distance)/(time)$] e. g. [$meter/second$]. Following that, we can say the vehicle's velocity is $x_n - x_{n-1}$[m/s](this is the abbreviation of meter/seconds) in this example. Let the velocity denote $v$[m/s] then the relation between the current position at $t$ second $x_t$ and initial position $x_0$ is
\begin{eqnarray}\label{xtx0vt}
	x_t = x_0 + v t .
\end{eqnarray}

\begin{figure}[H]
	\centering
	\includegraphics[width=0.5\textwidth]{images/accCar.png}
	\caption{constantly accelerating car}
	\label{cacar}
\end{figure}

Let's start the second case. Imagine a vehicle running on a line, and its speed increases constantly. Like the former example, let the speed at time $t$ be denoted as $v_t$.

Like the relation between distance and velocity in the former example, the increased speed between a single step is constant, and $v_t-v_{t-1}$ is an invariant for any natural number $t$. We call the increased amount acceleration. You may suppose the accerelation to have a unit $[(velocity)/(time)]$. Unfortunately, the supposed unit won't be used. Let's remind that velocity has the unit [$(distance)/(time)$], and put this into [$(velocity)/(time)$] then we earn commonly used unit [$(distance)/(time)^2$] e. g. [m/s$^2$].

We can find a relation between the velocity and the constant acceleration $a$, that is,
\begin{eqnarray}\label{vtv0at}
	v_t = v_0 + a \cdot t .
\end{eqnarray}

Next, we look at the relation between the traveled distance, the velocity, and the acceleration. 
Simply putting \ref{vtv0at} into \ref{xtx0vt} dosen't work because the result
\begin{eqnarray}
	x_t = x_0 + t(v_0 + at )
\end{eqnarray}
says that the car runs in the constant speed $v_0 + at$ from start measuring.

What we have to do is find a geometric relation between the position and the velocity. Let's revisit the first example and see eq.(\ref{xtx0vt}).

The traveled distance increases $v_0$ per second. Now, we consider a small timestep $\delta_t (\Delta_t << 1)$ and calculate the distance the vehicle traveled ($x << 1$ says that x is too small compared to 1).
We can easily get the answer
\begin{equation}
	v_0 \Delta_t \mathrm{[m]} .
\end{equation}
The answer can be seen as a rectangle whose width is $\Delta_t$ and height is $v_0$. Calculating total distance is interpreted as collecting such rectangles until the width being $t$.
The consequence is depicted in Fig.(\ref{geoDist}).
\begin{figure}[H]
	\centering
	\includegraphics[width=0.8\textwidth]{images/geoDist.png}
	\caption{Geometric aspect of the distance}
	\label{geoDist}
\end{figure}

Now, we expect that the total distance is the area of the graph whose x-axis is time and y-axis is velocity. In fact, the expectation is correct. I show this in a later section. For now, let the expectation be true.

\begin{figure}[H]
	\centering
	\includegraphics[width=0.4\textwidth]{images/x_v_aGraph.png}
	\caption{The graph of the second exapmle}
	\label{x_v_a}
\end{figure}

Return to the second example, and draw a graph of time vs velocity (Fig.(\ref{x_v_a})), then yield the correct total distance.
\begin{eqnarray}
	x_t = x_0 + v_0 t + \frac{1}{2} a t^2 .
\end{eqnarray}

\subsubsection{The Calculus}
We concluded that the relation between position, velocity, and acceleration can be interpreted geometrically in the last section. However, we can't calculate the distance whose graph has a curve. How do we do it?
\begin{figure}[H]
	\centering
	\includegraphics[width=0.6\textwidth]{images/vCurve.png}
	\caption{The graph with the velocity curve}
	\label{vCurve}
\end{figure}

First, as we did in the previous section, the area is assumed to be the sum of rectangles. We fix the width rather large and pick the height by matching the rectangle's right corner with the curve. The result will be the figure below.
\begin{figure}[H]
	\centering
	\includegraphics[width=0.6\textwidth]{images/vCurveLargeRec.png}
	\caption{The graph with the velocity curve with the large rectangles}
	\label{vCurveLR}
\end{figure}
We can see quite a large total error. This result is incorrect. Well then, let's see what happens when we set the width smaller.
\begin{figure}[H]
	\centering
	\includegraphics[width=0.6\textwidth]{images/vCurveSmallRec.png}
	\caption{The graph with the velocity curve with the small rectangles}
	\label{vCurveSR}
\end{figure}
While we didn't measure precisely, the total error looks smaller than before because they are split into small parts.

If this observation continues to the infinitesimal width, the total error also might go infinitesimal. The problem had been studied for a long time, and this manipulation is called "integration". As a consequence, it has concluded true.

However, the conclusion doesn't tell us how to calculate the total area. Thus next, we see how to calculate the sum briefly. 

The formulation is straightforward, summing the rectangles which have infinitesimal width.
Let the total area at a time $t$ $A(t)$, rectangles' width $dt$ and the velocity at time $t$ $v(t)$, $A(t)$ is
\begin{equation}
	A(t) = \sum_{i = 1}^{t/dt} v(i \cdot dt) \cdot dt ,
\end{equation}
and bring $dt$ to infinitesimal,
\begin{eqnarray}
	\label{integration}
	\lim_{dt \rightarrow 0} A(t) =\lim_{dt \rightarrow 0} \sum_{i=1}^{\infty}v(i \cdot dt) \cdot dt .
\end{eqnarray}

We can also write the summation in infinitely small rectangles like the notation below.
\begin{eqnarray}
	A(t) = \int_{0}^{t} f(t) dt ,
\end{eqnarray}
particularly, the relation between the total displacement $x$ and the velocity $v$ is written as
\begin{eqnarray}\label{xintv}
	x(t) = \int_{0}^{t} v(t) dt ,
\end{eqnarray}

Please note that this equation is merely formulating the above description.

Of course, we can calculate the integration from Eq.(\ref{integration}) using limit theory; many functions' integrals have already been solved by someone, and we can use the results instead. That's why I won't tell you details about the integration procedures. Here, you only have to remember the relation between the displacement and the velocity.

The same reasoning can be used for the relation between the velocity at $t$, $v(t)$, and the acceleration at $t$, $a(t)$; specifically,
\begin{equation}\label{vinta}
	v(t) = \int_{0}^{t} a(t) dt .
\end{equation}
Here, we can also derive the relation between the $x(t)$ and the $a(t)$ by inserting eq.(\ref{vinta}) into eq.(\ref{xintv}),
\begin{eqnarray}
	x(t) &=& \int_{0}^{t} v(t) dt\nonumber\\
		&=& \int_{0}^{t}(\int_{0}^{t}a(t')dt')dt .
\end{eqnarray}
Then, we can write this shortly as
\begin{equation}
		x(t) = \int_{0}^{t}\int_{0}^{t}a(t')dt'dt .
\end{equation}

Next, let's see the inverse relation: how to derive the velocity $v(t)$ from the displacement $x(t)$ (or acceleration $a(t)$ from the velocity).
\begin{figure}[H]
	\centering
	\includegraphics[width=0.6\textwidth]{images/VFromX.png}
	\caption{How do we derive $v(t)$ from $x(t)$?}
	\label{vfromx}
\end{figure}

The answer to the question, "How do we derive $v(t)$ from $x(t)$?" is to start by calculating the average velocity in an interval.
The average of the velocity between time $t$ and the time elapsed $\Delta t$, $t + \Delta t$, is an elementary calculation;
\begin{equation}
	(avg. velocity between t and t + \Delta t) = \frac{x(t+\Delta t)}{\Delta t} 
\end{equation}

However, what we want is the velocity at time $t$, then, like before, taking a small $\Delta t$, we get closer to the desired thing.
\begin{figure}[H]
	\centering
	\includegraphics[width=0.8\textwidth]{images/fDifferentialFromX.png}
	\caption{How do we derive $v(t)$ from $x(t)$?}
	\label{fDifferentialFromX}
\end{figure}

When the $\Delta t$ is infinitesimal (see fig. \ref{fDifferentialFromX}), the average of the velocity is identical to the velocity at $t$ (It's a bit paradoxical, but let this slide), that is, $v(t)$.

In mathematical terms, the above description can be written as
\begin{equation}
	v(t) = \lim_{\Delta t \rightarrow 0} \frac{x(t+\Delta t)}{\Delta t},
\end{equation}
or its abbreviation notation,
\begin{equation}\label{vdxdt}
v(t) = \frac{dx(t)}{dt}.
\end{equation}

The manipulation is called "differentiation".

Again, the relation between the acceleration at $t$, $a(t)$, and the velocity $v(t)$ is denoted as
\begin{equation}\label{advdt}
	a(t) = \frac{dv(t)}{dt} .
\end{equation}
Here, we can also derive the relation between the $a(t)$ and the $x(t)$ by inserting eq.(\ref{vdxdt}) into eq.(\ref{advdt}),
\begin{eqnarray}
	a(t) &=& \frac{dv(t)}{dt}\nonumber\\
	&=& \frac{\frac{dx(t)}{dt}}{dt}.
\end{eqnarray}
then we can write this as 
\begin{equation}
	a(t) = \frac{d^2 x(t)}{dt^2}.
\end{equation}

All relations are gathered eq. (\ref{relations}) for a recap(also see fig. \ref{xva}).
\begin{figure}[H]
	\centering
	\includegraphics[width=0.4\textwidth]{images/xva.png}
	\caption{The relations}
	\label{xva}
\end{figure}

\begin{eqnarray}\label{relations}
	x(t) &=& x(t) = \int_{0}^{t} v(t) dt ,\nonumber\\
	&=& \int_{0}^{t}\int_{0}^{t}a(t')dt'dt .\nonumber\\
	v(t) &=& \int_{0}^{t} a(t) dt\label{xvarelations}\\
	v(t) &=& \frac{dx(t)}{dt}\nonumber\\
	a(t) &=& \frac{dv(t)}{dt}\nonumber\\
	&=& \frac{d^2 x(t)}{dt^2}\nonumber.
\end{eqnarray}

\subsubsection{Newton's Laws of Motion}

In classical physics, there are principles to capture motion confirmed by experiments. They were named after the physicist who discovered them, Newton's Laws of Motion. Here, I emphasize "confirmed by experiments" because we have to admit them without proof. Once we admit them, we can construct various physical theories on them, and indeed, the theories match the natural effects.
Newton's Laws of Motion consist of three laws.
Then, let's see the laws.
\vspace{2truemm}

{\noindent\rm\bf\large Newton's First Law}

The first law is {\sl ``An object at rest remains at rest, and an object in motion remains in motion at constant speed and in a straight line unless acted on by an unbalanced force.''}. The law also interpreted as {\sl ``The velocity or the direction of the motion of the body is changed if the force acting on the body is unbalanced.''}

Let's imagine the rocket in space to understand the raw. Ideally, nothing affects it in space, and it keeps its speed without consuming fuel. When it lands on a star, it has to synchronize its speed with the star's, consuming fuel. When consuming fuel, the rocket serves force.

\begin{figure}[H]
	\centering
	\includegraphics[width=0.4\textwidth]{images/NewtonsFirstLaw.png}
	\caption{Newton's first Law}
	\label{NewF}
\end{figure}

On the other hand, everything around any planet is pulled to the planet's core. It is the gravity. We can stay still where we stand only when the gravity and the force from the land are balanced. It's also worth noting that any moving body on Earth receives opposite-direction damping force. The atmosphere around the planet causes this.\vspace{2truemm}

{\noindent\rm\bf\large Newton's Second Law}

The second law is {\sl The acceleration of an object depends on the mass of the object and the amount of force applied.}. In short, {\sl ``A heavier object is harder to move''}.

Look at Fig.(\ref{NewS}) with two different bodies whose mass are$M$ and $m$, are pushed the same force $\boldsymbol{F}$($\boldsymbol{F}$). If $M > m$, the body whose mass is $m$ has has a larger acceleration than the other.

\begin{figure}[H]
	\centering
	\includegraphics[width=0.4\textwidth]{images/NewtonsSecondLaw.png}
	\caption{Newton's second Law}
	\label{NewS}
\end{figure}

The second law also describes the rigorous relationship between force and acceleration in the formulation, 

\begin{equation}
	ma = F
\end{equation}

Where $m$ denotes the mass of the pushed body, $a$ equals the acceleration, and $\boldsymbol{F}$ is the force. The $F$'s unit is $Newton$ (or N) which is defined so that a 1[N] force gives a 1 [m/$\rm s^2$] acceleration to a 1 [kg] object.

Ofcource, The equation could be written with velocity $v$ or position $x$;

\begin{eqnarray}
	m\frac{dv}{dt} = F\\
	m\frac{d^2 x}{dt^2} = F
\end{eqnarray}

\vspace{2truemm}

{\noindent\rm\bf\large Newton's Third Law}

The third law is {\sl ``Whenever one object exerts a force on a second object, the second object exerts an equal and opposite force on the first.''}. Look at fig.(\ref{NewT}) about this law. Two people are receiving the same amount of force in opposite directions.

\begin{figure}[H]
	\centering
	\includegraphics[width=0.2\textwidth]{images/NewtonsThirdLaw.png}
	\caption{Newton's third Law}
	\label{NewT}
\end{figure}

I think there is no need to describe the law.

\subsubsection{Energy}\label{BasicEnergy}

Often, we want to know the total amount of force exerted.

\begin{figure}[H]
	\centering
	\includegraphics[width=0.4\textwidth]{images/constantForceEnergy.png}
	\caption{The energy produced by a constant force}
	\label{constForceEnergy}
\end{figure}

If the force $\boldsymbol{F}$ is constant, the quantity $W$ measured the below equation defined by the force$\boldsymbol{F}$ and moved distance $x$,

\begin{eqnarray}
	W = Fx.
\end{eqnarray}

\begin{figure}[H]
	\centering
	\includegraphics[width=0.4\textwidth]{images/ununiformForceEnergy.png}
	\caption{The energy produced by a varying force}
	\label{ununiformForceEnergy}
\end{figure}

If the force $F$ varies with its position(i. e. $F(x)$), the quantity produced by the force $F(x)$ from $x = x_s$ to $x = x_e$ is ,

\begin{equation}
	W = \int_{x_s}^{x_e}F(x) dx.\label{WintF}
\end{equation}

I skipped describing the detail because this is analogous to the relations among the displacement $\boldsymbol{x}$, the velocity $\boldsymbol{v}$, and the acceleration $\boldsymbol{a}$.

Next, let's see kinetic energy as an example of the energy.

First, recall that the force $F$ that exerts acceleration $a$ to a mass point whose mass $m$ is $F(x) = ma(x)$. Here, the values are scalar for simplicity.
Assigning this relation into \ref{WintF}.With given velocities $v(x_s) = v_s$ and $v(x_e) = v_e$, we have 

\begin{eqnarray}
		W &=& \int_{x_s}^{x_e}F(x) dx\\
		  &=& \int_{x_s}^{x_e}m\frac{dv}{dt}dx\\
		  &=& \int_{v_s}^{v_e}m\frac{dx}{dt}dv\\
		  &=&\int_{v_s}^{v_e}m vdv\\
		  &=&\left[\frac{1}{2}m v^2\right]_{v_s}^{v_e}.
\end{eqnarray}

In conclusion, the amount of the exerted work is $\frac{1}{2}m v_e^2-\frac{1}{2}mv_s^2$ if we have $v(x_s) = v_s$ and $v(x_e) = v_e$. The resulting expression $\frac{1}{2}m v^2$ is called kinetic energy.

Please note that the force has a direction in a general case because the acceleration has a direction. Therefore, the energy must be accumulated along the motion's path(but the integrated value is scalar). The calculation is called line integration. Actually, the added energy is the same regardless of the path if there is no dissipation force(e.g., friction). In other words, the energy can be derived if we have the original state and the resulting state.

\subsubsection{Summary}
First, this section describes the relation between position $x$, velocity $v$, and acceleration $a$. Next, we saw Newton's law of motion, which is vital to classical mechanics. Finally, we saw energy, the key to exerting force.

The basic physical simulations consist of these laws, but we have to do some additional work like sec. \ref{NumPhys} to put into practice.

\subsection{Constrained Dynamics}
Constrained Dynamics resolves physics problems that obey geometric constraints while satisfying Newton's law. In this dynamics, each of all constraints forms a constraint function that must meet a specific condition.
The functions take all related mass points' positions and are equal to zero if and only if the constraint is satisfied. From here, we concatenate all positions and denote as $\boldsymbol{x}$, and any constraint function is denoted as $C(\boldsymbol{x})$. Note that the function takes $\boldsymbol{x}$ but ignores the unrelated positions.

At least there are two styles of constrained dynamics. The first one is used in {\sl An Introduction to Physically Based Modeling: Constrained Dynamics} (\cite{ConstrainedWitkin}) and the other is used in {\sl Position Based Dynamics}(\cite{PBD}).

Although PBD uses the latter, I describe both of them for completeness. If you only want to understand PBD, you can skip section\ref{FBConstrained}.

\subsubsection{Force-Based Constrained Dynamics}
\label{FBConstrained}
Force-based constrained dynamics is start with a below constraint that is called {\sl Point-on-circle} constraint. Please imagine uniform circular motion with its center is the origin and its radii is $l$. Assume here the mass point rolling around has mass $m$ and its position is a 3d vector $\boldsymbol{x}$. 

\begin{figure}[H]
	\centering
	\includegraphics[width=0.2\textwidth]{images/pocConstraint.png}
	\caption{Point-on-circle constraint}
	\label{pocConstraint}
\end{figure}

\begin{equation}
	C(\boldsymbol{x}) = \frac{1}{2}(\boldsymbol{x}\cdot\boldsymbol{x}) - l^2
\end{equation}
Here, the dot between two vectors represents the dot product of the vectors.

Before we delve into the main topic, I provide a brief description of uniform circular motion.

If the mass point rotates $\omega [rad/s]$ on a plane whose third element is zero, its position is represented as

\begin{equation}
	\boldsymbol{x} = (\cos(\omega t),\sin(\omega t), 0).
\end{equation}

Next, we can get the velocity $\boldsymbol{v}$ and the acceleration, $\boldsymbol{a}$,

\begin{eqnarray}
	\boldsymbol{v} &=& \frac{d\boldsymbol{x}}{dt}\nonumber\\
	&=& \omega(-sin(\omega t), \cos(\omega t), 0),\\
	\boldsymbol{a} &=& \frac{d\boldsymbol{v}}{dt}\nonumber\\
	&=& \omega^2(-\cos(\omega t), -\sin(\omega t), 0)\nonumber\\
	&=& -\omega^2\boldsymbol{x}.
\end{eqnarray}

These three quantities have interesting relations described below.

\begin{eqnarray}
	& &\boldsymbol{x} \cdot  \boldsymbol{v} = 0\label{circularxv}\\
	& &\boldsymbol{a} \cdot  \boldsymbol{x} + \boldsymbol{v} \cdot \boldsymbol{v}= 0\label{circularaxvv}
\end{eqnarray}

Then, let's see the constraint function's properties.
If $C = 0$, the position $x$ is legal because $|\boldsymbol{x}| = l$.
Next, we try to get the first and second time derivatives of $C(\boldsymbol{x})$.

\begin{eqnarray}
	\frac{dC(\boldsymbol{x})}{dt} &=& \boldsymbol{x} \cdot \boldsymbol{v}\\
	\frac{d^2C(\boldsymbol{x})}{dt^2} &=&  \boldsymbol{a} \cdot \boldsymbol{x} + \boldsymbol{v} \cdot \boldsymbol{v}\label{dc2dt}
\end{eqnarray}
Here, we can use the relation of circular motion among quantities eq.\ref{circularxv} and \ref{circularaxvv}, then,

\begin{eqnarray}
	\frac{dC(\boldsymbol{x})}{dt} &=& 0\\
	\frac{d^2C(\boldsymbol{x})}{dt^2} &=& 0.\label{dc2dt0}
\end{eqnarray}

Thus, $\boldsymbol{x}$ is regal if $C(\boldsymbol{x}) = 0$, and $\boldsymbol{v}$ is regal if $\frac{dC(\boldsymbol{x})}{dt} = 0$, futhermore, $\boldsymbol{a}$ is regal if $\frac{d^2C(\boldsymbol{x})}{dt^2} = 0$. In particular, the last relation is heavily used in force-based constrained dynamics when solving the system.

Then, let's solve a constrained system with a mass point with the point-on-circle constraint whose setting is the same as described above.
Briefly, what we want to get is the force $\boldsymbol{\hat{f}}$ exerted by the constraint. We can write the equation that yields $\boldsymbol{\hat{f}}$ from the mass point's equation of motion,
\begin{eqnarray}
	\frac{d^2 x}{dt^2} = \frac{\boldsymbol{f}+\boldsymbol{\hat{f}}}{m}\label{EOMConstraiedDynamics}
\end{eqnarray}
Here, $\boldsymbol{f}$ is the external force that is irrelevant to the constraint, m is the mass of the point, and $\boldsymbol{x}$ is the mass point's position. We can substitute Eq. (\ref{EOMConstraiedDynamics})for Eq. (\ref{dc2dt}) and Eq. (\ref{dc2dt0}) if the mass point satisfies the legal position condition, the legal velocity condition, and the legal acceleration condition.

\begin{eqnarray}
	\frac{d^2 C(\boldsymbol{x})}{dt^2} = \frac{\boldsymbol{f}+\boldsymbol{\hat{f}}}{m}\cdot \boldsymbol{x} + \frac{d\boldsymbol{x}}{dt}\cdot\frac{d\boldsymbol{x}}{dt} = 0
\end{eqnarray}
 
Or, rewrite this and we get

\begin{eqnarray}
	\boldsymbol{\hat{f}} \cdot \boldsymbol{x} = -\boldsymbol{f}\cdot\boldsymbol{x} - m \frac{d\boldsymbol{x}}{dt}\cdot\frac{d\boldsymbol{x}}{dt}\label{fhatxeqmf}
\end{eqnarray}

The above equation has three unknowns, the elements of $\boldsymbol{\hat{f}}$. Nevertheless, we have only one equation. Thus, we have to get two additional conditions. We get the condition by requiring that the constraint force never add energy to nor remove energy from the system, i.e., the constraint is passive and lossless. We will use this condition in the kinetic energy $T(\boldsymbol{x})$ described in Sec. \ref{BasicEnergy},  $T(\boldsymbol{x}) = \frac{1}{2}m v^2$. The equation $T(\boldsymbol{x})$ differentiated by t is

\begin{eqnarray}
	\frac{dT(\boldsymbol{x})}{dt} &=& m\frac{d^2\boldsymbol{x}}{dt^2}\cdot \frac{d\boldsymbol{x}}{dt}\nonumber\\
	&=& \boldsymbol{f} \cdot \frac{d\boldsymbol{x}}{dt}+ \boldsymbol{\hat{f}}\cdot \frac{d\boldsymbol{x}}{dt}.
\end{eqnarray}

Here, using the fact that constraint force is passive and lossless, we can conclude that the contribution of the term $\boldsymbol{\hat{f}}\cdot \frac{d\boldsymbol{x}}{dt}$ doesn't attribute energy. Thus,

\begin{eqnarray}
	\boldsymbol{\hat{f}}\cdot \frac{d\boldsymbol{x}}{dt} = 0.\label{fhatxz}
\end{eqnarray}

In addition, the legal velocity condition holds $\boldsymbol{x} \cdot  \boldsymbol{v} = 0$. Using this and Eq. (\ref{fhatxz}), the constraint force $\boldsymbol{\hat{f}}$ is

\begin{equation}
	\boldsymbol{\hat{f}} = \lambda\boldsymbol{x},
\end{equation}

whose scalar $\lambda$ is unknown.

SSubstituting for $\boldsymbol{\hat{f}}$ in Eq. (\ref{fhatxeqmf}), and solving for $\lambda$ gives

\begin{equation}
	\lambda = \frac{-\boldsymbol{f}\cdot\boldsymbol{x} - m \frac{d\boldsymbol{x}}{dt}\cdot\frac{d\boldsymbol{x}}{dt}}{\boldsymbol{x} \cdot\boldsymbol{x}}.
\end{equation}

Having solved for $\lambda$, we can calculate $\boldsymbol{\hat{f}} = \lambda\boldsymbol{x}$, then get $\frac{d^2 \boldsymbol{x}}{dt^2} = \frac{\boldsymbol{f}+\boldsymbol{\hat{f}}}{m}$. After getting the acceleration, we can simulate the system by using it. The details of the simulation are described in Sec. \ref{NumPhys}.

A more complicated system with multiple constraints is described in Sec.  \ref{multipleConstraintsAssembly}.

\subsubsection{Position-Based Constrained Dynamics}
If we obey the position-based style, we can create a constrained function $C(\boldsymbol{x})$ by simply giving a function $\rho(\boldsymbol{x})$ that measures something among mass points, and an objective value $v$. Then, the function can be defined as
\begin{equation}
	(\boldsymbol{x}) \equiv \rho(\boldsymbol{x}) - v.
\end{equation}

Please remember that the constraint functions used in the last section were carefully constructed that satisfy three conditions.

We will see one of the simplest constraints in this section before learning the general case.

First, put two mass points $\boldsymbol{x}_1$ and $\boldsymbol{x}_2$ on the space and define a constraint function $C(\boldsymbol{x})$ which restrict two points' distance by $l$.
\begin{eqnarray}
	C(\boldsymbol{x}) \equiv | \boldsymbol{x}_1 - \boldsymbol{x}_2 | - l
\end{eqnarray}
For clarity, we use $C(\boldsymbol{x}_1,\boldsymbol{x}_2)$ instead of $C(\boldsymbol{x})$ from now.
If $C(\boldsymbol{x}_1,\boldsymbol{x}_2) \neq 0$ then 

\subsubsection{Multiple Constraints Assembly}\label{multipleConstraintsAssembly}
%force-basedにしろ、position-basedにしろ、行列を作れば解けるということを書く。

\inprep
\subsection{Continuum mechanics}
\inprep
\subsubsection{What is tensor?}
\inprep
\subsubsection{Force, strain, and stress}
\inprep
\subsubsection{Kirchhoff stress tensor}
\inprep
\subsubsection{Cauchy-Green deformation tensor}
\inprep
\subsubsection{St. Venant strain tensor}
\inprep

\section{Numerical physics}\label{NumPhys}
\inprep
\subsection{Verlet's integration}
\inprep
\subsection{Mass-spring system}
\inprep
\subsection{Explicit/implicit Euler method}
\inprep
\subsection{Shape matching}
\inprep

\section{Numerical integration}
\inprep
\subsection{The linear solvers}
\inprep
\subsection{Newton's method}
\inprep
\subsection{Iterative solvers}
\inprep
\subsection{Local/global solver}
\inprep
\subsection{Misc. topics}
\inprep
\subsubsection{Lagrange Multiplier}
\inprep
\subsubsection{LU decomposition}
\inprep
\subsubsection{Schur decomposition}
\inprep

\section{Column : In the terminology mess}
\inprep
\subsection{Newton*}
\inprep
\subsection{Euler*}
\inprep
\subsection{Jacobi*}
\inprep
\subsection{Lagrange*, Hamilton*, Hesse*}
\inprep

\section{Interplet the papers}
\inprep
\subsection{To PBD}
\inprep
\subsubsection{{\sl Large steps in cloth simulation}\cite{LargeStepBaraff}}
\inprep
\subsubsection{{\sl Advanced Character Physics}\cite{Jakobsen2003AdvancedCP}}
\inprep
\subsubsection{\small{\sl A Versatile and Robust Model for Geometrically Complex Deformable Solids}\cite{VersatileTeschner}}
\inprep
\subsubsection{{\sl Position Based Dynamics}\cite{PBD}}
\inprep

\subsection{To XPBD}
\inprep
\subsubsection{{\sl Geometric, Variational Integrators for Computer Animation}\cite{VariationalIntegrators2006}}
\inprep
\subsubsection{{\sl Interactive simulation of elastic deformable materials}\cite{Servin2006InteractiveSO}}
\inprep
\subsubsection{\small{\sl XPBD : position-based simulation of compliant constrained dynamics}\cite{XPBD}}
\inprep

\subsubsection{{\sl Small steps in physics simulation}\cite{SmallSteps}}
\inprep
\subsection{Bonus section : Projective Dynamics}
\inprep
\subsubsection{{\sl Example-based elastic materials}\cite{Example-basedMartin}}
\inprep
\subsubsection{{\sl Fast simulation of mass-spring systems}\cite{fastMassTiantian}}
\inprep
\subsubsection{\small{\sl Projective Dynamics: Fusing Constraint Projections for Fast Simulation}\cite{ProjDyn}}
\inprep
\subsubsection{\small{\sl ADMM $\subseteq$ projective dynamics: fast simulation of general constitutive models}\cite{ADMM_Proj}}
\inprep

\subsection{Post-XPBD}
\inprep
\newpage
\appendix




\section{Inportance of papers}
A lower number means more important. The papers' names are aligned in the lexicographic order as possible as I can.
\begin{enumerate}
	\item You must read these papers if you want to understand (X)PBD. But if you want to understand them completely, I recommend reading the others also.
		\begin{itemize}
			\item \pname{Position Based Dynamics}\cite{PBD} is one of the main subjects of this document.
			\newline
			\item \pname{XPBD : position-based simulation of compliant constrained dynamics}\cite{XPBD} addressed a numerical artifact that makes dependency between stiffness and \gls{iteration} count or size of the time-step.
		\end{itemize}
		\item  These papers are vital in understanding (X)PBD or have a strong impact on the field.
		\begin{itemize}
			\item\pname{Advanced Character Physics}\cite{Jakobsen2003AdvancedCP} introduced position based simulation method derived from Verlet's integration scheme and combined distance constraints. The concepts for PBD appear here, specifically, using distance and angular constraints, modifying vertices' positions directly, and solving \gls{constraint}s with some iterations. I recommend reading this before reading the PBD paper because this article is easier to read than PBD, contains insights into PBD, and presents source codes.
			\newline
			\item \pname{Large steps in cloth simulation}\cite{LargeStepBaraff} solved implicit integration with \gls{constraint}s by \gls{cgmethod} for off-line simulation. The scheme isn't used at PBD, but this one gives us a perspective of present physical simulations.
			\newline
			\item \pname{Projective Dynamics: Fusing Constraint Projections for Fast Simulation}\cite{ProjDyn} provided a position-based method that uses energy formulation and local/global solver. The method has a local Jacobi-like solver and a linear global equation one. The solvers enable robust and fast simulation without safeguards against singular or indefinite Hessians.
			If you want to understand state-of-the-art physics simulation methods, including PBD, it's better to read this.
			\newline
			\item \pname{ADMM $\subseteq$ projective dynamics: fast simulation of general constitutive models}\cite{ADMM_Proj} is worth reading because ADMM is actively researched now because of its robustness, parallelizability, and simplicity. 
		\end{itemize}
		\item These papers offer interesting discussions around PBD, deepen your understanding of PBD, or description of basic physical simulation scheme.
		\begin{itemize}
			\item \pname{Constraint Methods for Neural Networks and Computer Graphics}\cite{ConstrainedPlatt} describes the constraint methods for neural networks and computer graphics.
			It may not be easy to read because of 150 pages. But, if you have time, it's more worth reading this paper than some papers published before this one.
			\newline
			\item \pname{Example-based elastic materials}\cite{Example-basedMartin} provides a concept of the elastic manifold and optimization method for deforming into an artist-desirable state. However, this paper lacks reference to the manifold projection methods that are apparently well-known in the geometric optimization field.This paper helps us understand Projective dynamics, but I don't know Projective Dynamics's relevance with (X)PBD for now.
			\newline
			\item \pname{Fast simulation of mass-spring systems}\cite{fastMassTiantian} describes the local/global type solver to the mass-spring system clearly. Thus this paper helps us understand the variant solvers.
			\newline
			\item \pname{Interactive simulation of elastic deformable materials}\cite{Servin2006InteractiveSO} introduces physical parameters into constrained dynamics; the spirit is inherited by XPBD and there is an interesting discussion about the integrator. But I think this paper does explain the concept poorly,e. g. the integrator provided here, equation (20) lack of explanation, etc. 
			Therefore, this one classified here.
			\newline
			\item \pname{Nucleus: Towards a unified dynamics solver for computer graphics}\cite{Nucleus} is a good introduction to constrained dynamics because it shows its implementation aspect.
			However, it doesn't describe how constraints are resolved, so it isn't full-contained.
			\newline
			\item \pname{Physically Based Modeling: Principles and Practice}\cite{ConstrainedWitkin} is a course note about constrained dynamics. However, this note guides solving physical problems with constraints in a force-based style.
			\newline
			\item \pname{Robust treatment of collisions, contact and friction for cloth animation}\cite{RobustBridson2002}'s scheme separates physical simulation into internal parts and external parts. 
			Therefore, we can choose the internal modeling(e.g. mass-spring) and the external modeling(e.g. collision repulsion, friction, or gravity) independently. However, the scheme is not directly related to PBD.
			\newline
			\item \pname{Strain Based Dynamics}\cite{StrainBasedDyn}. The formulation of the strain tensor in XPBD seems to be based on 
			"Interactive Simulation of Elastic Deformable Materials"(2006) rather than this paper.
			However, the formulations described here are easy to understand if you are familiar with PBD and strain tensor.
			\newline
			\item \pname{Geometric, Variational Integrators for Computer Animation}\cite{VariationalIntegrators2006} presents the variational integrator from the Lagrangian/Hamiltonian physics that preserves linear/angular momenta.
			Its conservative quantities will be more important in fields such as robotics or accurate physical computation.
			But, this knowledge may be useless if you only want to understand  (X)PBD.
			\newline
			\item \pname{Efficient simulation of inextensible cloth}\cite{EfficientGoldenthai} treated the constrained system as globally linearized form and solved with a direct approach at each iteration.
		\end{itemize}
	\item These ones have historical value, but deeper discussions are done in other papers.
		\begin{itemize}
		\item \pname{Elastically deformable models}\cite{ElasticTerzopoulos}  brought the formulation of elastic bodies to computer graphics.
		\newline
		\item \pname{Energy Constraints On Parameterized Models}\cite{EnergyWitkin1987} uses shape representations that quite different from the current ones and the constraints presented in this paper are slightly inconvenient to the current ones. Thus, we no longer have to read this.
		\newline
		\item \pname{A modeling system based on dynamic constraints}\cite{Barzel1988} uses constraints as models' motion rather than to hold a model's detail and uses linear simultaneous equations when deriving forces.The points that are difficult to understand are that the paper doesn't describe the background of the equation derivation and that the symbols are scattered too much.
		Fortunately, we don't have to read this paper completely to understand present constrained dynamics because the style varies from the recent ones.
		\end{itemize}
	\item Not be classified yet.
		\begin{itemize}
			\item \pname{A Versatile and Robust Model for Geometrically Complex Deformable Solids}\cite{VersatileTeschner}
			\item \pname{Meshless deformations based on shape matching}\cite{MeshlessMuller2005}
			\item \pname{Fast Simulation of Inextensible Hair and Fur}\cite{InextensibleHair}
			\item \pname{Long Range Attachments - A Method to Simulate Inextensible Clothing in Computer Games}\cite{LongRangeAttachments}
			\item \pname{Position Based Fluids}\cite{PosBaseFluids}
			\item \pname{Position-based simulation of continuous materials}\cite{PBContimuousBENDER2014}
			\item \pname{Unified particle physics for real-time applications}\cite{UnifiedParticle}
			\item \pname{Air Meshes for Robust Collision Handling}\cite{AirMesh}
			\item \pname{A survey on position based dynamics, 2017}\cite{PBDCoursenote}
			\item \pname{Stable Constrained Dynamics}\cite{StableConstrainedDyn}
			\item \pname{Small steps in physics simulation}\cite{SmallSteps}
			\item \pname{Non-Smooth Newton Methods for Deformable Multi-Body Dynamics}\cite{NonSmooth}
			\item \pname{Detailed Rigid Body Simulation with Extended Position Based Dynamics}\cite{DetailedRigidbody}
			\item \pname{A Constraint-based Formulation of Stable Neo-Hookean Materials}\cite{ConstrainedBasedNeo-Hookean}
			\item \pname{Physically Based Shape Matching}\cite{PhysBaseShapeMatching}
			\item \pname{}\cite{}
			\item \pname{}\cite{}
			\item \pname{}\cite{}
			\item \pname{}\cite{}
			\item \pname{}\cite{}
			\item \pname{}\cite{}
			\item \pname{}\cite{}
			\item \pname{}\cite{}
			\item \pname{}\cite{}
			\item \pname{}\cite{}
			\item \pname{}\cite{}
			\item \pname{}\cite{}
			\item \pname{}\cite{}
			\item \pname{}\cite{}
			\item \pname{}\cite{}
			\item \pname{}\cite{}
			\item \pname{}\cite{}
			\item \pname{}\cite{}
			\item \pname{}\cite{}
			\item \pname{}\cite{}
			\item \pname{}\cite{}
			\item \pname{}\cite{}
			\item \pname{}\cite{}
		\end{itemize}
	
\end{enumerate}


\section{Glossaly}\label{ap1}
\subsection{symbols}

\subsection{terms}
\printglossary[nonumberlist, title={\hspace{-30pt}}, style=altlist]

\appendix

\addcontentsline{toc}{section}{References}
\bibliography{sn-bibliography}% common bib file


\end{document}


% \cite{bib1} 

%%==================%%
%% equation samples %%
%%==================%%
%\section{Equations}\label{sec4}
%$H\psi = E \psi$ is written via the command \verb+$H \psi = E \psi$+.

%\begin{equation}
%\|\tilde{X}(k)\|^2 \leq\frac{\sum\limits_{i=1}^{p}\left\|\tilde{Y}_i(k)\right\|^2+\sum\limits_{j=1}^{q}\left\|\tilde{Z}_j(k)\right\|^2 }{p+q}.\label{eq1}
%\end{equation}

%where,

%\begin{align}
%D_\mu &=  \partial_\mu - ig \frac{\lambda^a}{2} A^a_\mu \nonumber \\
%F^a_{\mu\nu} &= \partial_\mu A^a_\nu - \partial_\nu A^a_\mu + g f^{abc} A^b_\mu A^a_\nu \label{eq2}
%\end{align}

%\begin{equation}
%Y_\infty = \left( \frac{m}{\textrm{GeV}} \right)^{-3}
%    \left[ 1 + \frac{3 \ln(m/\textrm{GeV})}{15}
%    + \frac{\ln(c_2/5)}{15} \right]
%\end{equation}


%%==============%%
%%Table samples %%
%%==============%%

%\begin{table}[h]
%\caption{Caption text}\label{tab1}%
%\begin{tabular}{@{}llll@{}}
%\toprule
%Column 1 & Column 2  & Column 3 & Column 4\\
%\midrule
%row 1    & data 1   & data 2  & data 3  \\
%row 2    & data 4   & data 5\footnotemark[1]  & data 6  \\
%row 3    & data 7   & data 8  & data 9\footnotemark[2]  \\
%\botrule
%\end{tabular}
%\footnotetext{Source: This is an example of table footnote. This is an example of table footnote.}
%\footnotetext[1]{Example for a first table footnote. This is an example of table footnote.}
%\footnotetext[2]{Example for a second table footnote. This is an example of table footnote.}
%\end{table}

%===========================================================

%\begin{verbatim}
%\begin{table}[<placement-specifier>]
%\caption{<table-caption>}\label{<table-label>}%
%\begin{tabular}{@{}llll@{}}
%\toprule
%Column 1 & Column 2 & Column 3 & Column 4\\
%\midrule
%row 1 & data 1 & data 2	 & data 3 \\
%row 2 & data 4 & data 5\footnotemark[1] & data 6 \\
%row 3 & data 7 & data 8	 & data 9\footnotemark[2]\\
%\botrule
%\end{tabular}
%\footnotetext{Source: This is an example of table footnote. 
%This is an example of table footnote.}
%\footnotetext[1]{Example for a first table footnote.
%This is an example of table footnote.}
%\footnotetext[2]{Example for a second table footnote. 
%This is an example of table footnote.}
%\end{table}
%\end{verbatim}

%===========================================================

%\begin{table}[h]
%\caption{Example of a lengthy table which is set to full textwidth}\label{tab2}
%\begin{tabular*}{\textwidth}{@{\extracolsep\fill}lcccccc}
%\toprule%
%& \multicolumn{3}{@{}c@{}}{Element 1\footnotemark[1]} & \multicolumn{3}{@{}c@{}}{Element 2\footnotemark[2]} \\\cmidrule{2-4}\cmidrule{5-7}%
%Project & Energy & $\sigma_{calc}$ & $\sigma_{expt}$ & Energy & $\sigma_{calc}$ & $\sigma_{expt}$ \\
%\midrule
%Element 3  & 990 A & 1168 & $1547\pm12$ & 780 A & 1166 & $1239\pm100$\\
%Element 4  & 500 A & 961  & $922\pm10$  & 900 A & 1268 & $1092\pm40$\\
%\botrule
%\end{tabular*}
%\footnotetext{Note: This is an example of table footnote. This is an example of table footnote this is an example of table footnote this is an example of~table footnote this is an example of table footnote.}
%\footnotetext[1]{Example for a first table footnote.}
%\footnotetext[2]{Example for a second table footnote.}
%\end{table}

%===========================================================

%\begin{sidewaystable}
%\caption{Tables which are too long to fit, should be written using the ``sidewaystable'' environment as shown here}\label{tab3}
%\begin{tabular*}{\textheight}{@{\extracolsep\fill}lcccccc}
%\toprule%
%& \multicolumn{3}{@{}c@{}}{Element 1\footnotemark[1]}& \multicolumn{3}{@{}c@{}}{Element\footnotemark[2]} \\\cmidrule{2-4}\cmidrule{5-7}%
%Projectile & Energy	& $\sigma_{calc}$ & $\sigma_{expt}$ & Energy & $\sigma_{calc}$ & $\sigma_{expt}$ \\
%\midrule
%Element 3 & 990 A & 1168 & $1547\pm12$ & 780 A & 1166 & $1239\pm100$ \\
%Element 4 & 500 A & 961  & $922\pm10$  & 900 A & 1268 & $1092\pm40$ \\
%Element 5 & 990 A & 1168 & $1547\pm12$ & 780 A & 1166 & $1239\pm100$ \\
%Element 6 & 500 A & 961  & $922\pm10$  & 900 A & 1268 & $1092\pm40$ \\
%\botrule
%\end{tabular*}
%\footnotetext{Note: This is an example of table footnote this is an example of table footnote this is an example of table footnote this is an example of~table footnote this is an example of table footnote.}
%\footnotetext[1]{This is an example of table footnote.}
%\end{sidewaystable}


%%================%%
%%a figure sample %%
%%================%%
%\begin{figure}[h]
%\centering
%\includegraphics[width=0.9\textwidth]{fig.eps}
%\caption{This is a widefig. This is an example of long caption this is an example of long caption  this is an example of long caption this is an example of long caption}\label{fig1}
%\end{figure}


%%==============================================%%
%%Algorithms, Program codes and Listings samples%%
%%==============================================%%


%\lstset{texcl=true,basicstyle=\small\sf,commentstyle=\small\rm,mathescape=true,escapeinside={(*}{*)}}
%\begin{lstlisting}
%begin
%  for $i:=1$ to $10$ step $1$ do
%      expt($2,i$);  
%      newline() od                (*\textrm{Comments will be set flush to the right margin}*)
%where
%proc expt($x,n$) $\equiv$
%  $z:=1$;
%  do if $n=0$ then exit fi;
%     do if odd($n$) then exit fi;                 
%        comment: (*\textrm{This is a comment statement;}*)
%        $n:=n/2$; $x:=x*x$ od;
%     { $n>0$ };
%     $n:=n-1$; $z:=z*x$ od;
%  print($z$). 
%end
%\end{lstlisting}

%===========================================================

%
%\begin{algorithm}
%\caption{Calculate $y = x^n$}\label{algo1}
%\begin{algorithmic}[1]
%\Require $n \geq 0 \vee x \neq 0$
%\Ensure $y = x^n$ 
%\State $y \Leftarrow 1$
%\If{$n < 0$}\label{algln2}
%        \State $X \Leftarrow 1 / x$
%        \State $N \Leftarrow -n$
%\Else
%        \State $X \Leftarrow x$
%        \State $N \Leftarrow n$
%\EndIf
%\While{$N \neq 0$}
%        \If{$N$ is even}
%            \State $X \Leftarrow X \times X$
%            \State $N \Leftarrow N / 2$
%        \Else[$N$ is odd]
%            \State $y \Leftarrow y \times X$
%            \State $N \Leftarrow N - 1$
%        \EndIf
%\EndWhile
%\end{algorithmic}
%\end{algorithm}

%===========================================================

%\begin{minipage}{\hsize}%
%\lstset{frame=single,framexleftmargin=-1pt,framexrightmargin=-17pt,framesep=12pt,linewidth=0.98\textwidth,language=pascal}% Set your language (you can change the language for each code-block optionally)
%%%% Start your code-block
%\begin{lstlisting}
%for i:=maxint to 0 do
%begin
%{ do nothing }
%end;
%Write('Case insensitive ');
%Write('Pascal keywords.');
%\end{lstlisting}
%\end{minipage}

%%==================%%
%% citation samples %%
%%==================%%

%Here is an example for \verb+\cite{...}+: \cite{bib1}. Another example for \verb+\citep{...}+: \citep{bib2}. For author-year citation mode, \verb+\cite{...}+ prints Jones et al. (1990) and \verb+\citep{...}+ prints (Jones et al., 1990).

%===========================================================

%All cited bib entries are printed at the end of this article: \cite{bib3}, \cite{bib4}, \cite{bib5}, \cite{bib6}, \cite{bib7}, \cite{bib8}, \cite{bib9}, \cite{bib10}, \cite{bib11}, \cite{bib12} and \cite{bib13}.

%%====================%%
%% mathematic samples %%
%%====================%%

%\begin{tabular}{|l|p{19pc}|}
%\hline
%\verb+thmstyleone+ & Numbered, theorem head in bold font and theorem text in italic style \\\hline
%\verb+thmstyletwo+ & Numbered, theorem head in roman font and theorem text in italic style \\\hline
%\verb+thmstylethree+ & Numbered, theorem head in bold font and theorem text in roman style \\\hline
%\end{tabular}

%===========================================================

%\begin{theorem}[Theorem subhead]\label{thm1}
%\end{theorem}
%
%\begin{proposition}
%\end{proposition}
%
%\begin{example}
%\end{example}
%
%
%\begin{remark}
%\end{remark}
%
%\begin{definition}[Definition sub head]
%\end{definition}
%
%
%\begin{proof}
%\end{proof}
%
%
%\begin{proof}[Proof of Theorem~{\upshape\ref{thm1}}]
%\end{proof}
%
%\begin{quote}
%\end{quote}

 %(\url{https://www.nature.com/nature-research/editorial-policies}) for Nature Portfolio journals, 

\backmatter

%\bmhead{Supplementary information}
%If your article 

